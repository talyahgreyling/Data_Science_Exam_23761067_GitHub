\documentclass[preprint]{elsarticle}

\usepackage{lmodern}
%%%% My spacing
\usepackage{setspace}
\DeclareMathSizes{12}{14}{10}{10}

% Wrap around which gives all figures included the [H] command, or places it "here". This can be tedious to code in Rmarkdown.
\usepackage{float}
\let\origfigure\figure
\let\endorigfigure\endfigure
\renewenvironment{figure}[1][2] {
    \expandafter\origfigure\expandafter[H]
} {
    \endorigfigure
}

\let\origtable\table
\let\endorigtable\endtable
\renewenvironment{table}[1][2] {
    \expandafter\origtable\expandafter[H]
} {
    \endorigtable
}


\usepackage{ifxetex,ifluatex}
\usepackage{fixltx2e} % provides \textsubscript
\ifnum 0\ifxetex 1\fi\ifluatex 1\fi=0 % if pdftex
  \usepackage[T1]{fontenc}
  \usepackage[utf8]{inputenc}
\else % if luatex or xelatex
  \ifxetex
    \usepackage{mathspec}
    \usepackage{xltxtra,xunicode}
  \else
    \usepackage{fontspec}
  \fi
  \defaultfontfeatures{Mapping=tex-text,Scale=MatchLowercase}
  \newcommand{\euro}{€}
\fi

\usepackage{amssymb, amsmath, amsthm, amsfonts}

\def\bibsection{\section*{References}} %%% Make "References" appear before bibliography


\usepackage[numbers]{natbib}

\usepackage{longtable}
\usepackage[margin=cm,bottom=cm,top=cm, includefoot]{geometry}
\usepackage{fancyhdr}
\usepackage[bottom, hang, flushmargin]{footmisc}
\usepackage{graphicx}
\numberwithin{equation}{section}
\numberwithin{figure}{section}
\numberwithin{table}{section}
\setlength{\parindent}{0cm}
\setlength{\parskip}{1.3ex plus 0.5ex minus 0.3ex}
\usepackage{textcomp}
\renewcommand{\headrulewidth}{0pt}

\usepackage{array}
\newcolumntype{x}[1]{>{\centering\arraybackslash\hspace{0pt}}p{#1}}

%%%%  Remove the "preprint submitted to" part. Don't worry about this either, it just looks better without it:

 \def\tightlist{} % This allows for subbullets!

\usepackage{hyperref}
\hypersetup{breaklinks=true,
            bookmarks=true,
            colorlinks=true,
            citecolor=blue,
            urlcolor=blue,
            linkcolor=blue,
            pdfborder={0 0 0}}


% The following packages allow huxtable to work:
\usepackage{siunitx}
\usepackage{multirow}
\usepackage{hhline}
\usepackage{calc}
\usepackage{tabularx}
\usepackage{booktabs}
\usepackage{caption}


\newenvironment{columns}[1][]{}{}

\newenvironment{column}[1]{\begin{minipage}{#1}\ignorespaces}{%
\end{minipage}
\ifhmode\unskip\fi
\aftergroup\useignorespacesandallpars}

\def\useignorespacesandallpars#1\ignorespaces\fi{%
#1\fi\ignorespacesandallpars}

\makeatletter
\def\ignorespacesandallpars{%
  \@ifnextchar\par
    {\expandafter\ignorespacesandallpars\@gobble}%
    {}%
}
\makeatother


% definitions for citeproc citations
\NewDocumentCommand\citeproctext{}{}
\NewDocumentCommand\citeproc{mm}{%
\href{\#cite.\detokenize{#1}}{#2}\nocite{#1}}

\makeatletter
% allow citations to break across lines
\let\@cite@ofmt\@firstofone
% avoid brackets around text for \cite:
\def\@biblabel#1{}
\def\@cite#1#2{{#1\if@tempswa , #2\fi}}
\makeatother
\newlength{\cslhangindent}
\setlength{\cslhangindent}{1.5em}
\newlength{\csllabelwidth}
\setlength{\csllabelwidth}{3em}
\newenvironment{CSLReferences}[2] % #1 hanging-indent, #2 entry-spacing
{\begin{list}{}{%
	\setlength{\itemindent}{0pt}
	\setlength{\leftmargin}{0pt}
	\setlength{\parsep}{0pt}
	% turn on hanging indent if param 1 is 1
	\ifodd #1
	\setlength{\leftmargin}{\cslhangindent}
	\setlength{\itemindent}{-1\cslhangindent}
	\fi
	% set entry spacing
	\setlength{\itemsep}{#2\baselineskip}}}
{\end{list}}

\usepackage{calc}
\newcommand{\CSLBlock}[1]{\hfill\break\parbox[t]{\linewidth}{\strut\ignorespaces#1\strut}}
\newcommand{\CSLLeftMargin}[1]{\parbox[t]{\csllabelwidth}{\strut#1\strut}}
\newcommand{\CSLRightInline}[1]{\parbox[t]{\linewidth - \csllabelwidth}{\strut#1\strut}}
\newcommand{\CSLIndent}[1]{\hspace{\cslhangindent}#1}


\urlstyle{same}  % don't use monospace font for urls
\setlength{\parindent}{0pt}
\setlength{\parskip}{6pt plus 2pt minus 1pt}
\setlength{\emergencystretch}{3em}  % prevent overfull lines
\setcounter{secnumdepth}{0}

%%% Use protect on footnotes to avoid problems with footnotes in titles
\let\rmarkdownfootnote\footnote%
\def\footnote{\protect\rmarkdownfootnote}
\IfFileExists{upquote.sty}{\usepackage{upquote}}{}

%%% Include extra packages specified by user



\begin{document}



\begin{frontmatter}  %

\title{Booming Billions: an Analysis on Global Wealth Data}

% Set to FALSE if wanting to remove title (for submission)

\begin{abstract}
\small{
This pdf document contains an analysis on a collection of global
billionaires using data supplied by Forbes to uncover the underlying
patterns to becoming one of the world's wealthiest. Do developing
countries deliver more self-made men? Is the software industry the key
to a comfortable life? Read this paper to find out.
}
\end{abstract}

\vspace{1cm}





\vspace{0.5cm}

\end{frontmatter}

\setcounter{footnote}{0}



%________________________
% Header and Footers
%%%%%%%%%%%%%%%%%%%%%%%%%%%%%%%%%
\pagestyle{fancy}
\chead{}
\rhead{}
\lfoot{}
\rfoot{}
\lhead{}
%\rfoot{\footnotesize Page \thepage } % "e.g. Page 2"
\cfoot{}

%\setlength\headheight{30pt}
%%%%%%%%%%%%%%%%%%%%%%%%%%%%%%%%%
%________________________

\headsep 35pt % So that header does not go over title




\section{Introduction}\label{introduction}

Global wealth inequality has reached unprecedented levels, with just 8\%
of the world's population owning 85\% of its wealth and most of this
being concentrated in the top 1\% (Piketty 2015). Against this backdrop,
Forbes surveyed South Africa's wealthiest individuals and identified a
pressing need to analyze shifting wealth patterns across three
decades---from the 1990s to the mid-2010s. With disparities widening
both within and between nations (Mittelman 2007), this research aims to
uncover critical insights that could inform future economic strategies.
If successful, Forbes plans to expand this database, offering a vital
tool to address one of today's most urgent challenges: an increasingly
unequal world.

\subsection{Exploration Goals}\label{exploration-goals}

Two main claims (made by one of Forbs' participants) will be explored in
this document: (1) ``In the US, you saw an increasing number of new
billionaires emerge that had little to no familial ties to generational
wealth. Other developed markets and emerging markets tend to have less
entrepreneurial successes and tend to house mostly inherited wealth.''
(2) ``Most new self-made millionaires are in software, compared to
consumer services type industries in the 90s. This is related to
different countries' GDP, of course, with richer countries providing
more innovation in consumer services.''

\section{Exploring claim number 1}\label{exploring-claim-number-1}

\begin{itemize}
\tightlist
\item
  This claim entails comparing the prevalence of self-made billionaires
  in the US versus inherited wealth dominance in other developed and
  emerging markets.
\end{itemize}

\begin{figure}[H]

{\centering \includegraphics{Question4_files/figure-latex/unnamed-chunk-1-1} 

}

\caption{New vs. inherited wealth per decade (US)}\label{fig:unnamed-chunk-1}
\end{figure}

Figure \ref{Figure1} depicts the counts of US billionaires per decade
who become rich by inheritance (indicated by the pink bars) or their own
hard work (indicated by the blue bars). We see that in the US the amount
of self-made billionaires increases over the decades and outnumbers the
amount of billionaires by inheritance. We can therefore confirm the
first part of the first claim that an increasing number of new
billionaires emerged over the decades studies that had little to no
familial ties to generational wealth.

\begin{figure}[H]

{\centering \includegraphics{Question4_files/figure-latex/unnamed-chunk-2-1} 

}

\caption{New vs. inherited wealth per decade (other countries)}\label{fig:unnamed-chunk-2}
\end{figure}

Figure \ref{Figure2} depicts the counts of billionaires from other
developed markets and emerging markets per decade who become rich by
inheritance (indicated by the pink bars) or their own hard work
(indicated by the blue bars). We see that in other countries the amount
of self-made billionaires reduces slightly and then increases by a great
magnitude, outnumbering the amount of billionaires by inheritance by a
large margin. The billionaires by inheritance follow the same trajectory
with a lesser increase in the final decade recorded. We therefore find
the second part of the first claim, that these other markets tend to
have less entrepreneurial successes and house mostly inherited wealth,
to be false.

\section{Exploring claim number 2}\label{exploring-claim-number-2}

\begin{itemize}
\tightlist
\item
  Analyze the shift from consumer services (1990s) to software as the
  primary industry for self-made millionaires, correlating with national
  GDP levels.
\end{itemize}

\section{Additional insights}\label{additional-insights}

\begin{itemize}
\tightlist
\item
\end{itemize}

\phantomsection\label{refs}
\begin{CSLReferences}{1}{0}
\bibitem[\citeproctext]{ref-Mittelman2007}
Mittelman, James H. 2007. {``Paul Collier: The Bottom Billion: Why the
Poorest Countries Are Failing and What Can Be Done about It.''}
\emph{Population and Development Review} 33 (4): 821--21.

\bibitem[\citeproctext]{ref-Piketty2015}
Piketty, Thomas. 2015. {``Putting Distribution Back at the Center of
Economics: Reflections on "Capital in the Twenty-First Century".''}
\emph{The Journal of Economic Perspectives} 29 (1): 67--88.

\end{CSLReferences}

\bibliography{Tex/ref}





\end{document}
